\chapter*{Webservice Specification}
This section describes the Webservice API used to interact with LOST.

\section*{Overview}
LOST will provide a RESTful interface for automated interaction with other OSNAP systems. Human resources systems are expected to access and modify LOST user account information. The procurements department will also integrate with LOST to add and change the classification of various products. Only specific application instances should be allowed to use the REST interface to make changes to LOST data.

A strategy involving public/private cryptography will be used to authenticate interactions via the API. Public and private key files will be configured in the LOST configuration file. Most API calls will take in two arguments:
\begin{enumerate}
\item A blob containing the needed arguments, encrypted using the LOST public key
\item A cryptographic signature for the blob, signed using the sender's private key
\end{enumerate}
The signature is checked to verify a correct application is the sender. The argument blob is then decrypted. To provide some protection against replay, the request argument blob includes a timestamp. Timestamps will be given in UTC time and requests differing from the current time by more than 5 seconds should cause the API call to fail the request. LOST responses will include the timestamp provided in the request rather than the timestamp of when the response was generated.

The LOST web services will be located under /rest in the LOST URI space. When /rest is queried directly, it should return an HTML document describing the provided REST functions. Each API call expects to be called using the POST method with two named fields; the `arguments' field will contain the request JSON encrypted using the LOST public key and the `signature' field will contain a cryptographic signature of the encrypted message using the requester's private key. The response will be encrypted with the receiver's public key.

\section*{API Calls}
\subsection*{lost\_key}
This is the only lost call that does not authenticate the requester and uses plaintext input and output.
\\
\subsubsection*{Request -}
The request does not provide values for any of the arguments though the data and signature fields must be present in the request.

\subsubsection*{Response -}
\begin{description}
\item[result] `OK' or `FAIL'
\item[key] The LOST public key
\end{description}

\noindent Example:
\begin{verbatim}
{
    "timestamp": "2017-02-02 06:15:13",
    "result": "OK",
    "key": "bksaoudu......aoelchsauh"
}
\end{verbatim}



\subsection*{activate\_user}
Reactivates LOST access for a user or generates a new user account if needed.
\\

\subsubsection*{Request -}
\begin{description}
\item[username] OSNAP username to activate
\end{description}

\noindent Example:
\begin{verbatim}
{
    "timestamp": "2017-02-02 06:15:13",
    "username": "carter"
}
\end{verbatim}


\subsubsection*{Response -}
\begin{description}
\item[result] `OK', `NEW', or `FAIL'
\end{description}

\noindent Example:
\begin{verbatim}
{
    "timestamp": "2017-02-02 06:15:13",
    "result": "OK"
}
\end{verbatim}


\subsection*{suspend\_user}
Revokes access for a user. If the user does not exist or access has already been revoked, this call has no effect.
\\

\subsubsection*{Request -}
\begin{description}
\item[username] OSNAP username to revoke
\end{description}

\noindent Example:
\begin{verbatim}
{
    "timestamp": "2017-02-02 06:15:13",
    "username": "jackson"
}
\end{verbatim}


\subsubsection*{Response -}
\begin{description}
\item[result] `OK'
\end{description}

\noindent Example:
\begin{verbatim}
{
    "timestamp": "2017-02-02 06:15:13",
    "result": "OK"
}
\end{verbatim}


\subsection*{list\_products}
Requests a listing of all products in LOST based on a filter criteria. Additional filters may be provided and additional data may be returned. The requests of the minimum form documented in the example must be accepted and the produced result must have at least the structure and fields of example provided. Asset information may not be provided by this call (e.g. no returning number of assets that are of product types).
\\

\subsubsection*{Request -}
\begin{description}
\item[vendor] Case insensitive string to match against vendor name
\item[description] Case insensitive string to match against product description
\item[compartments] Security tags for the asset as a json array, all tags must be matched for compartment and level.
\end{description}

\noindent Example:
\begin{verbatim}
{
    "timestamp": "2017-02-02 06:15:13",
    "vendor": "",
    "description": "notepad",
    "compartments": []
}
\end{verbatim}

\subsubsection*{Response -}
\begin{description}
\item[listing] a json list containing the product information entries
\end{description}

\noindent Example:
\begin{verbatim}
{
    "timestamp": "2017-02-02 06:15:13",
    "listing": [
        {
            "vendor": "Dunder Mifflin",
            "description": "LOST legal size notepad",
            "compartments": []
        },
        {
            "vendor": "big n large",
            "description": "LOST legal size notepad"
            "compartments": []
        }
    ]
}
\end{verbatim}

\subsection*{add\_products}
Adds products to LOST. Requests should be atomic (all products are added or no products are added) and attempting to add a duplicate product should cause the request to fail. A product is duplicated if at the end of the call the database would have two products with the same vendor and description.
\\

\subsubsection*{Request -}
\begin{description}
\item[new\_products] a json list describing the new products
\end{description}

\noindent Example:
\begin{verbatim}
{
    "timestamp": "2017-02-02 06:15:13",
    "new_products": [
        {
            "vendor": "Dunder Mifflin",
            "description": "LOST letter size notepad",
            "alt_description": "Children's storybook"
            "compartments": ["adm:s"]
        },
        {
            "vendor": "Stark Industries",
            "description": "Micronized Arc Reactor",
            "alt_description": "Baseball"
            "compartments": ["nrg:ts","wpn:s"]
        }
}
\end{verbatim}

\subsubsection*{Response -}
\begin{description}
\item[result] `OK', or `FAIL'
\end{description}

\noindent Example:
\begin{verbatim}
{
    "timestamp": "2017-02-02 06:15:13",
    "result": "OK"
}
\end{verbatim}


\subsection*{add\_asset}
Adds a new asset to LOST.
\\
\subsubsection*{Request -}
\begin{description}
\item[vendor] Vendor of the product
\item[description] Description of the product
\item[compartments] Additional compartments the asset should be classified under
\item[facility] Facility code the asset is initially housed at
\end{description}

\noindent Example:
\begin{verbatim}
{
    "timestamp": "2017-02-02 06:15:13",
    "vendor": "Dunder Mifflin",
    "description": "LOST letter size notepad",
    "compartments": ["wpn:ts"],
    "facility": "HQ"
}
\end{verbatim}


\subsubsection*{Response -}
\begin{description}
\item[result] `OK', or `FAIL'
\end{description}

\noindent Example:
\begin{verbatim}
{
    "timestamp": "2017-02-02 06:15:13",
    "result": "OK"
}
\end{verbatim}